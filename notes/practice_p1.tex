\section{Practice Problem 1:}
\begin{itemize}
  \item On each visit the robot spews out one candy with probability $0.6$, two candies with probability 0.3 or three with probability 0.1
  \item Let Q(U=u) be the probability distribution function for the robot spewing the candy.
  \begin{itemize}
    \item Q(u=1) = 0.6
    \item Q(u=2) = 0.3
    \item Q(u=3) = 0.1
  \end{itemize}
  \item To refill the candy compartment, it can estimate tan accurate count with probability 0.6 or it can be off by one in either direction with probability
  \item Let P(X=x) to be the probability distribution function for the robots measurement of how many candy its holding.
  \begin{itemize}
    \item P(x=x) = 0.6
    \item P(x=x+1) = 0.2
    \item P(x=x-1) = 0.2
  \end{itemize}
\end{itemize}

\textbf{We start with $50$ candies, After the first visit, sensor measured $48$.
After the second visit, the sensor measured $48$ again. Calculate the belief of the number of candy in the comporatement after the first and the second visit.
}

\begin{align}
  p(x \mid z) &= \frac{p(z \mid x)p(x)}{p(z)} \\
  p(x) &= \frac{p(x \mid z)p(z)}{p(z | x)} \\
  \label{eq:Bayes Rule}
\end{align}

\begin{align}
  P(u = 1) &= 0.6 \\
  P(u = 2) &= 0.3 \\
  P(u = 3) &= 0.1 \\
  \label{eq: Action Probability Distributions}
\end{align}

Model the state as:
\begin{equation}
  x(t+1) = x(t) - u(t) \\
  \label{eq:State Model}
\end{equation}

Initially, we have a belief of the number of candy in the comporatement to be: $bel(x_t=50) = 1.0$
At the first visit, the robot takes an action. Now, we have:
\begin{align*}
  \bar{bel}(x_t=49) &= 0.6 \\
  \bar{bel}(x_t=48) &= 0.3 \\
  \bar{bel}(x_t=47) &= 0.1 \\
\end{align*}

Here we get a measurement from our sensor, and we measure $z = 48$. But since our measurement model has uncertainty, this belief might be wrong.
We need to calculate the bayes update to get a posterior. So we can do:
\begin{equation}
  P(z=48 \mid x_t) = \eta P(z=48 \mid x=49)
  P(z=48 \mid x=48) = 0.6
  P(z=48 \mid x=47) = 0.2
  \label{eq:}
\end{equation}

% # we get measurement 48
%
% # if we had u=1, x=49: 0.6 * 0.2 = 0.12
% # if we had u=2, x=48: 0.3 * 0.6 = 0.18
% # if we had u=3, x=47: 0.1 * 0.2 = 0.02
% unnormalized_posterior = np.array([[0.12], [0.18], [0.02]])
% posterior = unnormalized_posterior / sum(unnormalized_posterior)
%
% # [0.375  0.5625 0.0625] means
% # [49 48 47]
%
% """
% if we had x_t-1 = 49, u=1: x=48 0.375 * 0.6
%                       u=2: x=47 0.375 * 0.3
%                       u=3: x=46 0.375 * 0.1
%
% if we had x_t-1 = 48, u=1: x=47 0.5625 * 0.6
%                       u=2: x=46 0.5625 * 0.3
%                       u=3: x=45 0.5625 * 0.1
%
% if we had x_t-1 = 47, u=1: x=46 0.0625 * 0.6
%                       u=2: x=45 0.0625 * 0.3
%                       u=3: x=44 0.0625 * 0.1
%
%     we got measurement z=48 again
%
% Which means we could have gotten P(z=48 | x=48): (0.375 * 0.6) * 0.6
%                                  P(z=48 | x=47): (0.375 * 0.3) * 0.2 + (0.5625 * 0.6) * 0.2
%
% """
% # Take action
% bel_48 = 0.6 * 0.375
% bel_47 = 0.375 * 0.3 + 0.5626 * 0.6
% bel_46 = 0.375 * 0.1 + 0.5625 * 0.3 + 0.0625 * 0.6
% bel_45 = 0.5625 * 0.1 + 0.0625 * 0.3
% bel_44 = 0.0625 * 0.1
% bel_bar = np.array([[bel_48], [bel_47], [bel_46], [bel_45], [bel_44]])
% unnormalized_posterior = np.array([[bel_48], [bel_47], [bel_46], [bel_45], [bel_44]])
%
% measurement_model = np.array([0.6, 0.2, 0.0, 0.0, 0.0])
% bel_bar = np.array([0.225, 0.450, 0.244, 0.075, 0.006])
% unnormalized_posterior = bel_bar * measurement_model
% posterior = unnormalized_posterior / sum(unnormalized_posterior)
%
% print(f"{posterior}")
% # bel = [z=x+1, z=x, z=x-1]
% measurement_mdl = np.array([[0.2, 0.6, 0.2]])
%
% After visit 1 (z=48): bel(x(1)) = [0.375, 0.5625, 0.0625] for x ∈ {49, 48, 47}
% After visit 2 (z=48): bel(x(2)) = [0.6, 0.4, 0.0, 0.0, 0.0] for x ∈ {48, 47, 46, 45, 44}


