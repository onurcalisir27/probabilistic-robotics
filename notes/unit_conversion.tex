
\section{Unit Conversion Tables}

\begin{tcolorbox}[colback=red!10!white,colframe=red!75!black,title= EXAM TIP: Always Convert to SI First!]
Before doing ANY calculation:
\begin{enumerate}
    \item Convert ALL inputs to SI units
    \item Do your calculation in SI
    \item Convert output back if needed
\end{enumerate}

\textbf{Common exam trap:} Mixing units (e.g., velocity in mph with time in seconds)
\end{tcolorbox}

% ============================================================================
\subsection{Length \& Distance}
% ============================================================================

\begin{table}[H]
\centering
\begin{tabular}{|l|l|l|}
\hline
\rowcolor{blue!20}
\textbf{Unit} & \textbf{Symbol} & \textbf{Conversion to meters (m)} \\
\hline
Millimeter & mm & $\times 10^{-3}$ or $\div 1000$ \\
\hline
Centimeter & cm & $\times 10^{-2}$ or $\div 100$ \\
\hline
Kilometer & km & $\times 10^{3}$ or $\times 1000$ \\
\hline
\rowcolor{yellow!20}
Inch & in & $\times 0.0254$ or $\times 2.54 \times 10^{-2}$ \\
\hline
\rowcolor{yellow!20}
Foot & ft & $\times 0.3048$ \\
\hline
\rowcolor{yellow!20}
Yard & yd & $\times 0.9144$ \\
\hline
\rowcolor{yellow!20}
Mile & mi & $\times 1609.34$ \\
\hline
\rowcolor{yellow!20}
Nautical mile & nmi & $\times 1852$ \\
\hline
\end{tabular}
\caption{Length conversions (yellow = Imperial/US)}
\end{table}

\textbf{Common robotics values:}
\begin{itemize}
    \item Wheel radius: 33 mm = 0.033 m = 3.3 cm
    \item Robot width: 143.5 mm = 0.1435 m = 14.35 cm
    \item Camera height: 6 ft = 1.829 m
\end{itemize}

% ============================================================================
\subsection{Mass}
% ============================================================================

\begin{table}[H]
\centering
\begin{tabular}{|l|l|l|}
\hline
\rowcolor{blue!20}
\textbf{Unit} & \textbf{Symbol} & \textbf{Conversion to kilograms (kg)} \\
\hline
Gram & g & $\times 10^{-3}$ or $\div 1000$ \\
\hline
Milligram & mg & $\times 10^{-6}$ or $\div 1{,}000{,}000$ \\
\hline
Metric ton & t & $\times 1000$ \\
\hline
\rowcolor{yellow!20}
Pound (mass) & lbm & $\times 0.453592$ \\
\hline
\rowcolor{yellow!20}
Ounce & oz & $\times 0.0283495$ \\
\hline
\rowcolor{yellow!20}
Slug & slug & $\times 14.5939$ \\
\hline
\end{tabular}
\caption{Mass conversions}
\end{table}

\textbf{Quick approximations:}
\begin{itemize}
    \item 1 kg $\approx$ 2.2 lbm
    \item 1 lbm $\approx$ 0.45 kg
\end{itemize}

% ============================================================================
\subsection{Force}
% ============================================================================

\begin{table}[H]
\centering
\begin{tabular}{|l|l|l|}
\hline
\rowcolor{blue!20}
\textbf{Unit} & \textbf{Symbol} & \textbf{Conversion to Newtons (N)} \\
\hline
Kilonewton & kN & $\times 1000$ \\
\hline
Millinewton & mN & $\times 10^{-3}$ or $\div 1000$ \\
\hline
Dyne & dyn & $\times 10^{-5}$ \\
\hline
\rowcolor{yellow!20}
Pound-force & lbf & $\times 4.44822$ \\
\hline
\rowcolor{yellow!20}
Kilogram-force & kgf & $\times 9.80665$ \\
\hline
\rowcolor{yellow!20}
Ounce-force & ozf & $\times 0.278014$ \\
\hline
\end{tabular}
\caption{Force conversions}
\end{table}

\textbf{Weight conversions (assuming $g = 9.81$ m/s²):}
\begin{itemize}
    \item Weight of 1 kg mass: $F = mg = 1 \times 9.81 = 9.81$ N
    \item 1 lbf $\approx$ 4.45 N (weight of 1 lbm mass on Earth)
\end{itemize}

\begin{tcolorbox}[colback=orange!10!white,colframe=orange!75!black,title= Common Confusion: lbm vs lbf]
\begin{itemize}
    \item \textbf{lbm} (pound mass) = unit of mass = 0.454 kg
    \item \textbf{lbf} (pound force) = unit of force = 4.45 N
    \item Relationship: $F[\text{lbf}] = m[\text{lbm}] \times g[\text{ft/s}^2] / 32.174$
    \item On Earth: 1 lbm weighs 1 lbf (by design)
\end{itemize}
\end{tcolorbox}

% ============================================================================
\subsection{Linear Velocity}
% ============================================================================

\begin{table}[H]
\centering
\begin{tabular}{|l|l|l|}
\hline
\rowcolor{blue!20}
\textbf{Unit} & \textbf{Symbol} & \textbf{Conversion to m/s} \\
\hline
Kilometer/hour & km/h & $\div 3.6$ or $\times 0.277778$ \\
\hline
Centimeter/second & cm/s & $\times 0.01$ or $\div 100$ \\
\hline
Millimeter/second & mm/s & $\times 0.001$ or $\div 1000$ \\
\hline
\rowcolor{yellow!20}
Feet/second & ft/s & $\times 0.3048$ \\
\hline
\rowcolor{yellow!20}
Miles/hour & mph & $\times 0.44704$ \\
\hline
\rowcolor{yellow!20}
Knots & kn or kt & $\times 0.514444$ \\
\hline
\rowcolor{yellow!20}
Inch/second & in/s & $\times 0.0254$ \\
\hline
\end{tabular}
\caption{Linear velocity conversions}
\end{table}

\textbf{Quick approximations:}
\begin{itemize}
    \item 100 km/h $\approx$ 28 m/s
    \item 60 mph $\approx$ 27 m/s (approximately 100 km/h)
    \item 1 m/s $\approx$ 3.6 km/h
\end{itemize}

% ============================================================================
\subsection{Linear Acceleration}
% ============================================================================

\begin{table}[H]
\centering
\begin{tabular}{|l|l|l|}
\hline
\rowcolor{blue!20}
\textbf{Unit} & \textbf{Symbol} & \textbf{Conversion to m/s²} \\
\hline
Standard gravity & $g$ & $\times 9.80665$ (exact) or $\approx 9.81$ \\
\hline
Centimeter/second² & cm/s² & $\times 0.01$ or $\div 100$ \\
\hline
Gal (galileo) & Gal & $\times 0.01$ (= 1 cm/s²) \\
\hline
\rowcolor{yellow!20}
Feet/second² & ft/s² & $\times 0.3048$ \\
\hline
\rowcolor{yellow!20}
Inch/second² & in/s² & $\times 0.0254$ \\
\hline
\end{tabular}
\caption{Linear acceleration conversions}
\end{table}

\textbf{Common accelerations:}
\begin{itemize}
    \item Earth gravity: $g = 9.81$ m/s² $= 32.174$ ft/s²
    \item Fast car: 0-60 mph in 3 sec $\approx 8.9$ m/s² $\approx 0.9g$
    \item Elevator: $\approx 1-2$ m/s²
\end{itemize}

% ============================================================================
\subsection{Angular Measure}
% ============================================================================

\begin{table}[H]
\centering
\begin{tabular}{|l|l|l|}
\hline
\rowcolor{blue!20}
\textbf{Unit} & \textbf{Symbol} & \textbf{Conversion to radians (rad)} \\
\hline
Degree & ° or deg & $\times \frac{\pi}{180} \approx 0.0174533$ \\
\hline
Revolution & rev & $\times 2\pi \approx 6.28319$ \\
\hline
Gradian & grad & $\times \frac{\pi}{200}$ \\
\hline
Arcminute & ' & $\times \frac{\pi}{10800}$ \\
\hline
Arcsecond & '' & $\times \frac{\pi}{648000}$ \\
\hline
\end{tabular}
\caption{Angular measure conversions}
\end{table}

\textbf{Key conversions (memorize these):}
\begin{align*}
180 &= \pi \text{ rad} \\
90 &= \frac{\pi}{2} \text{ rad} \approx 1.5708 \text{ rad} \\
45 &= \frac{\pi}{4} \text{ rad} \approx 0.7854 \text{ rad} \\
30 &= \frac{\pi}{6} \text{ rad} \approx 0.5236 \text{ rad} \\
1 &\approx 0.01745 \text{ rad} \\
1 \text{ rad} &\approx 57.2958
\end{align*}

% ============================================================================
\subsection{Angular Velocity}
% ============================================================================

\begin{table}[H]
\centering
\begin{tabular}{|l|l|l|}
\hline
\rowcolor{blue!20}
\textbf{Unit} & \textbf{Symbol} & \textbf{Conversion to rad/s} \\
\hline
Degree/second & deg/s or °/s & $\times \frac{\pi}{180} \approx 0.0174533$ \\
\hline
Revolution/second & rev/s or rps & $\times 2\pi \approx 6.28319$ \\
\hline
Revolution/minute & rpm & $\times \frac{2\pi}{60} \approx 0.10472$ \\
\hline
Hertz (cycles/sec) & Hz & $\times 2\pi \approx 6.28319$ \\
\hline
\end{tabular}
\caption{Angular velocity conversions}
\end{table}

\textbf{Common values:}
\begin{itemize}
    \item Motor: 3000 rpm = 314.16 rad/s $\approx$ 50 rev/s
    \item Wheel encoder: 100 rpm = 10.47 rad/s
    \item Earth rotation: 1 rev/day = $7.27 \times 10^{-5}$ rad/s
\end{itemize}

\textbf{Quick formula:}
\begin{equation}
\omega[\text{rad/s}] = \frac{2\pi \cdot N[\text{rpm}]}{60} = 0.10472 \cdot N[\text{rpm}]
\end{equation}

% ============================================================================
\subsection{Time}
% ============================================================================

\begin{table}[H]
\centering
\begin{tabular}{|l|l|l|}
\hline
\rowcolor{blue!20}
\textbf{Unit} & \textbf{Symbol} & \textbf{Conversion to seconds (s)} \\
\hline
Millisecond & ms & $\times 10^{-3}$ or $\div 1000$ \\
\hline
Microsecond & $\mu$s & $\times 10^{-6}$ or $\div 1{,}000{,}000$ \\
\hline
Nanosecond & ns & $\times 10^{-9}$ \\
\hline
Minute & min & $\times 60$ \\
\hline
Hour & h or hr & $\times 3600$ \\
\hline
Day & d & $\times 86400$ \\
\hline
\end{tabular}
\caption{Time conversions}
\end{table}

\textbf{Sampling rates:}
\begin{itemize}
    \item 100 Hz = 0.01 s = 10 ms period
    \item 50 Hz = 0.02 s = 20 ms period
    \item 10 Hz = 0.1 s = 100 ms period
    \item 1 kHz = 0.001 s = 1 ms period
\end{itemize}

% ============================================================================
\subsection{Area}
% ============================================================================

\begin{table}[H]
\centering
\begin{tabular}{|l|l|l|}
\hline
\rowcolor{blue!20}
\textbf{Unit} & \textbf{Symbol} & \textbf{Conversion to m²} \\
\hline
Square centimeter & cm² & $\times 10^{-4}$ or $\div 10{,}000$ \\
\hline
Square millimeter & mm² & $\times 10^{-6}$ \\
\hline
Square kilometer & km² & $\times 10^{6}$ \\
\hline
Hectare & ha & $\times 10{,}000$ \\
\hline
\rowcolor{yellow!20}
Square inch & in² & $\times 0.00064516$ \\
\hline
\rowcolor{yellow!20}
Square foot & ft² & $\times 0.092903$ \\
\hline
\rowcolor{yellow!20}
Square yard & yd² & $\times 0.836127$ \\
\hline
\rowcolor{yellow!20}
Acre & ac & $\times 4046.86$ \\
\hline
\end{tabular}
\caption{Area conversions}
\end{table}

% ============================================================================
\subsection{Volume}
% ============================================================================

\begin{table}[H]
\centering
\begin{tabular}{|l|l|l|}
\hline
\rowcolor{blue!20}
\textbf{Unit} & \textbf{Symbol} & \textbf{Conversion to m³} \\
\hline
Liter & L & $\times 10^{-3}$ or $\div 1000$ \\
\hline
Milliliter & mL & $\times 10^{-6}$ \\
\hline
Cubic centimeter & cm³ or cc & $\times 10^{-6}$ (= 1 mL) \\
\hline
Cubic millimeter & mm³ & $\times 10^{-9}$ \\
\hline
\rowcolor{yellow!20}
Cubic inch & in³ & $\times 1.6387 \times 10^{-5}$ \\
\hline
\rowcolor{yellow!20}
Cubic foot & ft³ & $\times 0.0283168$ \\
\hline
\rowcolor{yellow!20}
US gallon & gal & $\times 0.00378541$ \\
\hline
\rowcolor{yellow!20}
US fluid ounce & fl oz & $\times 2.9574 \times 10^{-5}$ \\
\hline
\end{tabular}
\caption{Volume conversions}
\end{table}

% ============================================================================
\subsection{Density}
% ============================================================================

\begin{table}[H]
\centering
\begin{tabular}{|l|l|l|}
\hline
\rowcolor{blue!20}
\textbf{Unit} & \textbf{Symbol} & \textbf{Conversion to kg/m³} \\
\hline
Gram/cm³ & g/cm³ & $\times 1000$ \\
\hline
Gram/liter & g/L & (already kg/m³) \\
\hline
\rowcolor{yellow!20}
Pound/cubic foot & lb/ft³ & $\times 16.0185$ \\
\hline
\rowcolor{yellow!20}
Pound/cubic inch & lb/in³ & $\times 27{,}679.9$ \\
\hline
\rowcolor{yellow!20}
Slug/cubic foot & slug/ft³ & $\times 515.379$ \\
\hline
\end{tabular}
\caption{Density conversions}
\end{table}

\textbf{Common densities:}
\begin{itemize}
    \item Water: 1000 kg/m³ = 1 g/cm³
    \item Air (STP): 1.225 kg/m³
    \item Steel: $\approx$ 7850 kg/m³
    \item Aluminum: $\approx$ 2700 kg/m³
\end{itemize}

% ============================================================================
\subsection{Pressure \& Stress}
% ============================================================================

\begin{table}[H]
\centering
\begin{tabular}{|l|l|l|}
\hline
\rowcolor{blue!20}
\textbf{Unit} & \textbf{Symbol} & \textbf{Conversion to Pascals (Pa)} \\
\hline
Kilopascal & kPa & $\times 1000$ \\
\hline
Megapascal & MPa & $\times 10^{6}$ \\
\hline
Bar & bar & $\times 10^{5}$ or $\times 100{,}000$ \\
\hline
Millibar & mbar & $\times 100$ \\
\hline
Atmosphere & atm & $\times 101{,}325$ \\
\hline
\rowcolor{yellow!20}
Pound/square inch & psi & $\times 6894.76$ \\
\hline
\rowcolor{yellow!20}
Pound/square foot & psf & $\times 47.8803$ \\
\hline
Torr (mmHg) & Torr & $\times 133.322$ \\
\hline
\end{tabular}
\caption{Pressure conversions}
\end{table}

\textbf{Common pressures:}
\begin{itemize}
    \item Atmospheric: 1 atm = 101.325 kPa = 14.7 psi
    \item Tire pressure: 30 psi $\approx$ 207 kPa $\approx$ 2.07 bar
\end{itemize}

% ============================================================================
\subsection{Energy \& Work}
% ============================================================================

\begin{table}[H]
\centering
\begin{tabular}{|l|l|l|}
\hline
\rowcolor{blue!20}
\textbf{Unit} & \textbf{Symbol} & \textbf{Conversion to Joules (J)} \\
\hline
Kilojoule & kJ & $\times 1000$ \\
\hline
Watt-hour & Wh & $\times 3600$ \\
\hline
Kilowatt-hour & kWh & $\times 3.6 \times 10^{6}$ \\
\hline
Calorie & cal & $\times 4.184$ \\
\hline
Kilocalorie & kcal & $\times 4184$ \\
\hline
Electronvolt & eV & $\times 1.602 \times 10^{-19}$ \\
\hline
\rowcolor{yellow!20}
British thermal unit & BTU & $\times 1055.06$ \\
\hline
\rowcolor{yellow!20}
Foot-pound & ft·lbf & $\times 1.35582$ \\
\hline
\rowcolor{yellow!20}
Inch-pound & in·lbf & $\times 0.112985$ \\
\hline
\end{tabular}
\caption{Energy conversions}
\end{table}

\textbf{Note:} $1 \text{ J} = 1 \text{ N·m} = 1 \text{ kg·m}^2\text{/s}^2$

% ============================================================================
\subsection{Power}
% ============================================================================

\begin{table}[H]
\centering
\begin{tabular}{|l|l|l|}
\hline
\rowcolor{blue!20}
\textbf{Unit} & \textbf{Symbol} & \textbf{Conversion to Watts (W)} \\
\hline
Kilowatt & kW & $\times 1000$ \\
\hline
Megawatt & MW & $\times 10^{6}$ \\
\hline
Milliwatt & mW & $\times 10^{-3}$ or $\div 1000$ \\
\hline
\rowcolor{yellow!20}
Horsepower (mech.) & hp & $\times 745.7$ \\
\hline
\rowcolor{yellow!20}
Horsepower (metric) & PS & $\times 735.5$ \\
\hline
\rowcolor{yellow!20}
Foot-pound/second & ft·lbf/s & $\times 1.35582$ \\
\hline
BTU/hour & BTU/h & $\times 0.293071$ \\
\hline
\end{tabular}
\caption{Power conversions}
\end{table}

\textbf{Common powers:}
\begin{itemize}
    \item Small DC motor: 10-100 W
    \item Car engine: 100-300 hp $\approx$ 75-225 kW
    \item Human sustained: $\approx$ 75 W (1/10 hp)
\end{itemize}

\textbf{Note:} $1 \text{ W} = 1 \text{ J/s} = 1 \text{ N·m/s}$

% ============================================================================
\subsection{Torque}
% ============================================================================

\begin{table}[H]
\centering
\begin{tabular}{|l|l|l|}
\hline
\rowcolor{blue!20}
\textbf{Unit} & \textbf{Symbol} & \textbf{Conversion to N·m} \\
\hline
Kilonewton-meter & kN·m & $\times 1000$ \\
\hline
Newton-centimeter & N·cm & $\times 0.01$ or $\div 100$ \\
\hline
Newton-millimeter & N·mm & $\times 0.001$ or $\div 1000$ \\
\hline
\rowcolor{yellow!20}
Pound-foot & lb·ft or lbf·ft & $\times 1.35582$ \\
\hline
\rowcolor{yellow!20}
Pound-inch & lb·in or lbf·in & $\times 0.112985$ \\
\hline
\rowcolor{yellow!20}
Ounce-inch & oz·in & $\times 0.00706155$ \\
\hline
\end{tabular}
\caption{Torque conversions}
\end{table}

\textbf{Note:} Torque and energy have the same dimensions (N·m), but torque is NOT measured in Joules!

\begin{tcolorbox}[colback=orange!10!white,colframe=orange!75!black,title=Power-Torque-Speed Relationship]
\begin{equation}
P = \tau \cdot \omega
\end{equation}
where:
\begin{itemize}
    \item $P$ = power [W]
    \item $\tau$ = torque [N·m]
    \item $\omega$ = angular velocity [rad/s]
\end{itemize}

\textbf{Imperial version:}
\begin{equation}
\text{HP} = \frac{\tau[\text{lb·ft}] \times \text{RPM}}{5252}
\end{equation}
\end{tcolorbox}

% ============================================================================
\subsection{Moment of Inertia}
% ============================================================================

\begin{table}[H]
\centering
\begin{tabular}{|l|l|l|}
\hline
\rowcolor{blue!20}
\textbf{Unit} & \textbf{Symbol} & \textbf{Conversion to kg·m²} \\
\hline
Gram-cm² & g·cm² & $\times 10^{-7}$ \\
\hline
Kilogram-cm² & kg·cm² & $\times 10^{-4}$ or $\div 10{,}000$ \\
\hline
\rowcolor{yellow!20}
Pound-foot² & lb·ft² & $\times 0.0421401$ \\
\hline
\rowcolor{yellow!20}
Pound-inch² & lb·in² & $\times 2.9264 \times 10^{-4}$ \\
\hline
\rowcolor{yellow!20}
Slug-foot² & slug·ft² & $\times 1.35582$ \\
\hline
\end{tabular}
\caption{Moment of inertia conversions}
\end{table}

\textbf{Common shapes:}
\begin{itemize}
    \item Solid cylinder (about axis): $I = \frac{1}{2}mr^2$
    \item Solid sphere: $I = \frac{2}{5}mr^2$
    \item Thin rod (about center): $I = \frac{1}{12}mL^2$
    \item Point mass at distance $r$: $I = mr^2$
\end{itemize}

% ============================================================================
\subsection{Frequency \& Sampling Rate}
% ============================================================================

\begin{table}[H]
\centering
\begin{tabular}{|l|l|l|}
\hline
\rowcolor{blue!20}
\textbf{Unit} & \textbf{Symbol} & \textbf{Conversion to Hz} \\
\hline
Kilohertz & kHz & $\times 1000$ \\
\hline
Megahertz & MHz & $\times 10^{6}$ \\
\hline
Gigahertz & GHz & $\times 10^{9}$ \\
\hline
Cycles/minute & cpm & $\div 60$ \\
\hline
Revolutions/minute & rpm & $\div 60$ \\
\hline
\end{tabular}
\caption{Frequency conversions}
\end{table}

\textbf{Period-Frequency relationship:}
\begin{equation}
f[\text{Hz}] = \frac{1}{T[\text{s}]} \quad \Leftrightarrow \quad T[\text{s}] = \frac{1}{f[\text{Hz}]}
\end{equation}

\textbf{Common sampling rates:}
\begin{itemize}
    \item IMU: 100-1000 Hz ($\Delta t$ = 10-1 ms)
    \item Camera: 30-60 Hz ($\Delta t$ = 33-17 ms)
    \item LIDAR: 10-40 Hz ($\Delta t$ = 100-25 ms)
    \item Control loop: 50-500 Hz ($\Delta t$ = 20-2 ms)
\end{itemize}

% ============================================================================
% ============================================================================


