% ============================================================================
% ROBOT MOTION - EXAM NOTES
% Focus: Motion models (velocity & odometry), noise parameters
% ============================================================================

\section{Quick Reference: Robot Motion Models}

\begin{tcolorbox}[colback=yellow!10!white,colframe=orange!75!black,title=\textbf{Robot Motion Models - Fast Reference}]

\textbf{Pose Representation:} $x = \begin{bmatrix} x \\ y \\ \theta \end{bmatrix}$ (position + orientation)

\vspace{3mm}
\textbf{Two Motion Models:}

\begin{tabular}{|l|l|l|}
\hline
\textbf{Aspect} & \textbf{Velocity Model} & \textbf{Odometry Model} \\
\hline
Input $u_t$ & $(v_t, \omega_t)$ - commanded velocities & $(\bar{x}_{t-1}, \bar{x}_t)$ - poses from encoders \\
Accuracy & Less accurate & More accurate \\
Availability & Before motion (predictive) & After motion (retrospective) \\
Use case & Motion planning & State estimation \\
\hline
\end{tabular}

\vspace{3mm}
\textbf{Velocity Motion Model:}
\begin{align*}
u_t &= \begin{bmatrix} v \\ \omega \end{bmatrix}, \quad r = \left|\frac{v}{\omega}\right| \quad \text{(circular motion radius)} \\
x_t &= x_{t-1} + \begin{bmatrix}
-\frac{\hat{v}}{\hat{\omega}}\sin\theta + \frac{\hat{v}}{\hat{\omega}}\sin(\theta + \hat{\omega}\Delta t) \\
\frac{\hat{v}}{\hat{\omega}}\cos\theta - \frac{\hat{v}}{\hat{\omega}}\cos(\theta + \hat{\omega}\Delta t) \\
\hat{\omega}\Delta t
\end{bmatrix}
\end{align*}
where $\hat{v} = v + \epsilon_{\alpha_1 v^2 + \alpha_2 \omega^2}$, $\hat{\omega} = \omega + \epsilon_{\alpha_3 v^2 + \alpha_4 \omega^2}$

\vspace{3mm}
\textbf{Odometry Motion Model Decomposition:}
\begin{itemize}
    \item Rotation 1: $\delta_{rot1} = \arctan2(\bar{y}' - \bar{y}, \bar{x}' - \bar{x}) - \bar{\theta}$
    \item Translation: $\delta_{trans} = \sqrt{(\bar{x}' - \bar{x})^2 + (\bar{y}' - \bar{y})^2}$
    \item Rotation 2: $\delta_{rot2} = \bar{\theta}' - \bar{\theta} - \delta_{rot1}$
\end{itemize}

\vspace{3mm}
\textbf{Noise Parameters:}
\begin{itemize}
    \item Velocity: $\alpha_1, \alpha_2$ affect translation, $\alpha_3, \alpha_4$ affect rotation
    \item Odometry: $\alpha_1, \alpha_2$ affect rotations, $\alpha_3, \alpha_4$ affect translation
    \item Larger $\alpha$ = more uncertainty
\end{itemize}

\vspace{3mm}
\textbf{Probability Function:}
$$\text{prob}(a, b^2) = \frac{1}{\sqrt{2\pi b^2}} \exp\left\{-\frac{a^2}{2b^2}\right\} = \mathcal{N}(a; 0, b^2)$$

\end{tcolorbox}

% ============================================================================
\section{Robot Kinematics and Pose}
% ============================================================================

\subsection{Kinematic Configuration}

\textbf{Kinematics} describes how control actions affect robot configuration.

\textbf{For planar mobile robots:}
\begin{itemize}
    \item \textbf{Pose:} $x = \begin{bmatrix} x & y & \theta \end{bmatrix}^T$
    \begin{itemize}
        \item $(x, y)$ - 2D position in global frame
        \item $\theta$ - orientation (heading/bearing) relative to x-axis
    \end{itemize}
    \item \textbf{Location:} Pose without orientation = $(x, y)$
\end{itemize}

\begin{figure}[H]
  \begin{center}
    \includegraphics[width=0.4\textwidth]{images/robot_pose.png}
  \end{center}
  \caption{Robot pose in global coordinate system}
\end{figure}

\subsection{Probabilistic Motion Model}

The \textbf{motion model} specifies the state transition probability:
\begin{equation}
p(x_t \mid u_t, x_{t-1})
\end{equation}

This describes the probability distribution over poses $x_t$ that result from:
\begin{itemize}
    \item Starting at pose $x_{t-1}$
    \item Executing control $u_t$
\end{itemize}

\textbf{Key Insight:} Robot motion is inherently uncertain due to:
\begin{itemize}
    \item Wheel slippage
    \item Uneven terrain
    \item Control inaccuracies
    \item Systematic errors
\end{itemize}

% ============================================================================
\section{Velocity Motion Model}
% ============================================================================

\subsection{Overview}

\textbf{Control input:} $u_t = \begin{bmatrix} v_t \\ \omega_t \end{bmatrix}$
\begin{itemize}
    \item $v_t$ - translational (linear) velocity [m/s]
    \item $\omega_t$ - rotational (angular) velocity [rad/s]
    \item Positive $v$ = forward, positive $\omega$ = counterclockwise
\end{itemize}

\textbf{Use case:} When you have velocity commands (motion planning, prediction)

\subsection{Exact Kinematics (No Noise)}

For constant velocities $(v, \omega)$ over time interval $\Delta t$:

\textbf{Motion on a circle:}
\begin{equation}
r = \left|\frac{v}{\omega}\right| \quad \text{(radius of circular trajectory)}
\end{equation}

Special case: $\omega = 0 \Rightarrow r = \infty$ (straight line)

\textbf{Circle center:} Starting from $x_{t-1} = (x, y, \theta)^T$:
\begin{align}
x_c &= x - \frac{v}{\omega}\sin\theta \\
y_c &= y + \frac{v}{\omega}\cos\theta
\end{align}

\textbf{Final pose after $\Delta t$:}
\begin{equation}
\boxed{
\begin{bmatrix} x' \\ y' \\ \theta' \end{bmatrix} =
\begin{bmatrix} x \\ y \\ \theta \end{bmatrix} +
\begin{bmatrix}
-\frac{v}{\omega}\sin\theta + \frac{v}{\omega}\sin(\theta + \omega\Delta t) \\
\frac{v}{\omega}\cos\theta - \frac{v}{\omega}\cos(\theta + \omega\Delta t) \\
\omega\Delta t
\end{bmatrix}
}
\label{eq:ideal_velocity_motion}
\end{equation}

\begin{figure}[H]
\centering
\begin{tikzpicture}[scale=0.8]
    % Draw coordinate system
    \draw[->] (-0.5,0) -- (5,0) node[right] {$x$};
    \draw[->] (0,-0.5) -- (0,4) node[above] {$y$};

    % Draw circular arc
    \draw[thick, blue] (1,0.5) arc (-30:30:2);

    % Start pose
    \fill (1,0.5) circle (2pt);
    \draw[->, thick, red] (1,0.5) -- (1.3,0.7) node[anchor=west] {$x_{t-1}$};

    % End pose
    \fill (3.46,1.5) circle (2pt);
    \draw[->, thick, red] (3.46,1.5) -- (3.7,1.85) node[anchor=south] {$x_t$};

    % Center
    \fill[green!50!black] (2.5,2.5) circle (2pt) node[above right] {$(x_c, y_c)$};
    \draw[dashed, green!50!black] (2.5,2.5) -- (1,0.5) node[midway, above] {$r$};

    % Velocity arrows
    \draw[->, thick, purple] (1.5,0.8) arc (-20:0:0.5) node[midway, right] {$\omega$};
    \draw[->, thick, orange] (2,1) -- (2.3,1.3) node[midway, above] {$v$};

\end{tikzpicture}
\caption{Velocity motion model: robot follows circular arc}
\end{figure}

\subsection{Probabilistic Velocity Model (With Noise)}

Real robots have errors in executing velocity commands.

\textbf{Noise model:} Actual velocities differ from commanded:
\begin{align}
\hat{v} &= v + \epsilon_{\alpha_1 v^2 + \alpha_2 \omega^2} \\
\hat{\omega} &= \omega + \epsilon_{\alpha_3 v^2 + \alpha_4 \omega^2} \\
\hat{\gamma} &= \epsilon_{\alpha_5 v^2 + \alpha_6 \omega^2} \quad \text{(drift)}
\end{align}

where $\epsilon_{b^2}$ denotes a zero-mean Gaussian with variance $b^2$.

\textbf{Noise characteristics:}
\begin{itemize}
    \item Error magnitude proportional to velocity magnitudes
    \item $\alpha_1, \alpha_2$ - translational velocity error parameters
    \item $\alpha_3, \alpha_4$ - rotational velocity error parameters
    \item $\alpha_5, \alpha_6$ - drift error parameters
\end{itemize}

\subsection{Velocity Motion Model Algorithm}

\begin{algorithm}[H]
\caption{Velocity Motion Model: $p(x_t \mid u_t, x_{t-1})$}
\KwInput{$x_t = (x', y', \theta')^T$, $u_t = (v, \omega)^T$, $x_{t-1} = (x, y, \theta)^T$}
\KwOutput{Probability $p(x_t \mid u_t, x_{t-1})$}
\BlankLine
\tcp{Compute ideal (noise-free) control to reach $x_t$ from $x_{t-1}$}
\BlankLine
$\mu = \frac{1}{2}\left[\frac{(x - x')\cos \theta + (y - y') \sin \theta}{(y - y') \cos \theta - (x - x')\sin \theta}\right]$\;
\BlankLine
$x^* = \frac{x+x'}{2} + \mu (y-y')$\;
\BlankLine
$y^* = \frac{y+y'}{2} + \mu (x'-x)$\;
\BlankLine
$r^* = \sqrt{(x-x^*)^2 + (y-y^*)^2}$\;
\BlankLine
$\Delta \theta = \atantwo(y' - y^*,x'-x^*)- \atantwo(y-y^*, x-x^*)$\;
\BlankLine
$\hat{v} = \frac{\Delta \theta}{\dt} r^*$\;
\BlankLine
$\hat{\omega} = \frac{\Delta \theta}{\Delta t}$\;
\BlankLine
$\hat{\gamma} = \frac{\theta' - \theta}{\Delta t} - \hat{\omega}$\;

\BlankLine
\tcp{Compute probability based on difference from commanded velocities}
$p_1 = \text{prob}(v - \hat{v}, \alpha_1 v^2 + \alpha_2 \omega^2)$\;
$p_2 = \text{prob}(\omega - \hat{\omega}, \alpha_3 v^2 + \alpha_4 \omega^2)$\;
$p_3 = \text{prob}(\hat{\gamma}, \alpha_5 v^2 + \alpha_6 \omega^2)$\;

\BlankLine
\Return{$p_1 \cdot p_2 \cdot p_3$}
\end{algorithm}

\textbf{Intuition:}
\begin{enumerate}
    \item Given: start pose $x_{t-1}$, commanded $(v, \omega)$, hypothesized end pose $x_t$
    \item Compute: what velocities $(\hat{v}, \hat{\omega}, \hat{\gamma})$ would be needed to reach $x_t$?
    \item Return: probability that commanded velocities resulted in needed velocities
\end{enumerate}

\begin{figure}[H]
  \begin{center}
    \includegraphics[width=0.6\textwidth]{images/vel_motion_model_noise.png}
  \end{center}
  \caption{Velocity motion model for different noise parameters: (a) moderate noise, (b) large translational noise, (c) large rotational noise}
\end{figure}

% ============================================================================
\section{Odometry Motion Model}
% ============================================================================

\subsection{Overview}

\textbf{Control input:} $u_t = \begin{bmatrix} \bar{x}_{t-1} \\ \bar{x}_t \end{bmatrix}$ where $\bar{x} = (\bar{x}, \bar{y}, \bar{\theta})^T$
\begin{itemize}
    \item $\bar{x}_{t-1}$, $\bar{x}_t$ - poses from robot's internal odometry (wheel encoders)
    \item Bar notation indicates odometry frame (not global frame)
\end{itemize}

\textbf{Key idea:}
\begin{itemize}
    \item Robot's internal odometry frame doesn't match global frame
    \item But \textit{relative motion} in odometry frame approximates relative motion in global frame
    \item Use odometry readings to estimate motion, not absolute position
\end{itemize}

\textbf{Use case:} State estimation after motion has occurred (more accurate than velocity model)

\textbf{Why more accurate?}
\begin{itemize}
    \item Measures actual wheel rotations (not just commands)
    \item No model mismatch between commanded and achieved velocities
    \item Still suffers from drift and slippage (but less than velocity model)
\end{itemize}

\subsection{Odometry Decomposition}

Any planar motion can be decomposed into: \textbf{Rotate → Translate → Rotate}

\begin{figure}[H]
  \begin{center}
    \includegraphics[width=0.45\textwidth]{images/odom_model.png}
  \end{center}
  \caption{Odometry motion decomposition into three steps}
\end{figure}

\begin{tcolorbox}[colback=blue!5!white,colframe=blue!75!black,title=Odometry Decomposition Formulas]

Given odometry readings $\bar{x}_{t-1} = (\bar{x}, \bar{y}, \bar{\theta})$ and $\bar{x}_t = (\bar{x}', \bar{y}', \bar{\theta}')$:

\vspace{2mm}
\textbf{Step 1: Initial rotation} (align with direction of translation)
\begin{equation}
\delta_{rot1} = \arctan2(\bar{y}' - \bar{y}, \bar{x}' - \bar{x}) - \bar{\theta}
\end{equation}

\textbf{Step 2: Translation} (straight-line distance)
\begin{equation}
\delta_{trans} = \sqrt{(\bar{x}' - \bar{x})^2 + (\bar{y}' - \bar{y})^2}
\end{equation}

\textbf{Step 3: Final rotation} (achieve final orientation)
\begin{equation}
\delta_{rot2} = \bar{\theta}' - \bar{\theta} - \delta_{rot1}
\end{equation}

\textbf{Important:} All angles must be normalized to $[-\pi, \pi]$

\end{tcolorbox}

\subsection{Probabilistic Odometry Model}

Each of the three motion components ($\delta_{rot1}, \delta_{trans}, \delta_{rot2}$) is corrupted by noise:

\textbf{Noise model:}
\begin{align}
\delta_{rot1} &\rightarrow \delta_{rot1} + \epsilon_{\alpha_1 |\delta_{rot1}| + \alpha_2 \delta_{trans}} \\
\delta_{trans} &\rightarrow \delta_{trans} + \epsilon_{\alpha_3 \delta_{trans} + \alpha_4(|\delta_{rot1}| + |\delta_{rot2}|)} \\
\delta_{rot2} &\rightarrow \delta_{rot2} + \epsilon_{\alpha_1 |\delta_{rot2}| + \alpha_2 \delta_{trans}}
\end{align}

\textbf{Noise parameters:}
\begin{itemize}
    \item $\alpha_1$ - rotation error from rotation
    \item $\alpha_2$ - rotation error from translation
    \item $\alpha_3$ - translation error from translation
    \item $\alpha_4$ - translation error from rotation
\end{itemize}

\subsection{Odometry Motion Model Algorithm}

\begin{algorithm}[H]
\caption{Odometry Motion Model: $p(x_t \mid u_t, x_{t-1})$}
\KwInput{$x_t = (x', y', \theta')^T$, $u_t = (\bar{x}_{t-1}, \bar{x}_t)$, $x_{t-1} = (x, y, \theta)^T$}
\KwOutput{Probability $p(x_t \mid u_t, x_{t-1})$}

\BlankLine
\tcp{Extract relative motion from odometry readings}
$\delta_{rot1} = \arctan2(\bar{y}' - \bar{y}, \bar{x}' - \bar{x}) - \bar{\theta}$\;
$\delta_{trans} = \sqrt{(\bar{x}' - \bar{x})^2 + (\bar{y}' - \bar{y})^2}$\;
$\delta_{rot2} = \bar{\theta}' - \bar{\theta} - \delta_{rot1}$\;

\BlankLine
\tcp{Compute expected motion from true poses}
$\hat{\delta}_{rot1} = \arctan2(y' - y, x' - x) - \theta$\;
$\hat{\delta}_{trans} = \sqrt{(x' - x)^2 + (y' - y)^2}$\;
$\hat{\delta}_{rot2} = \theta' - \theta - \hat{\delta}_{rot1}$\;

\BlankLine
\tcp{Compute probability of each motion component}
$p_1 = \text{prob}(\delta_{rot1} - \hat{\delta}_{rot1}, \alpha_1 \hat{\delta}_{rot1}^2 + \alpha_2 \hat{\delta}_{trans}^2)$\;
$p_2 = \text{prob}(\delta_{trans} - \hat{\delta}_{trans}, \alpha_3 \hat{\delta}_{trans}^2 + \alpha_4 (\hat{\delta}_{rot1}^2 + \hat{\delta}_{rot2}^2))$\;
$p_3 = \text{prob}(\delta_{rot2} - \hat{\delta}_{rot2}, \alpha_1 \hat{\delta}_{rot2}^2 + \alpha_2 \hat{\delta}_{trans}^2)$\;

\BlankLine
\Return{$p_1 \cdot p_2 \cdot p_3$}
\end{algorithm}

\textbf{Intuition:}
\begin{enumerate}
    \item \textbf{From odometry:} Extract what motion the robot \textit{thinks} it did ($\delta$'s)
    \item \textbf{From true poses:} Compute what motion \textit{actually occurred} ($\hat{\delta}$'s)
    \item \textbf{Compare:} Probability based on difference (larger difference = less likely)
\end{enumerate}

\textbf{Critical implementation detail:} Normalize all angular differences to $[-\pi, \pi]$ before computing probabilities!

\begin{figure}[H]
  \begin{center}
    \includegraphics[width=0.65\textwidth]{images/odom_motion_model_noise.png}
  \end{center}
  \caption{Odometry motion model for different noise parameters: (a) typical noise, (b) large translational error, (c) large rotational error}
\end{figure}

% ============================================================================
\section{Comparison: Velocity vs. Odometry Models}
% ============================================================================

\begin{tcolorbox}[colback=green!5!white,colframe=green!60!black,title=Model Comparison]

\begin{tabular}{|l|p{5cm}|p{5cm}|}
\hline
\textbf{Property} & \textbf{Velocity Model} & \textbf{Odometry Model} \\
\hline
\textbf{Input} & Commanded velocities $(v, \omega)$ & Encoder-based poses $(\bar{x}_{t-1}, \bar{x}_t)$ \\
\hline
\textbf{Availability} & Before motion (predictive) & After motion (retrospective) \\
\hline
\textbf{Accuracy} & Lower (model mismatch) & Higher (measures actual motion) \\
\hline
\textbf{Suitable for} & Motion planning, prediction & State estimation, localization \\
\hline
\textbf{Error sources} &
$\bullet$ Control inaccuracy \newline
$\bullet$ Model mismatch \newline
$\bullet$ Slippage \newline
$\bullet$ Drift &
$\bullet$ Slippage \newline
$\bullet$ Drift \newline
$\bullet$ Encoder resolution \\
\hline
\textbf{Noise params} & $\alpha_1$ to $\alpha_6$ & $\alpha_1$ to $\alpha_4$ \\
\hline
\textbf{Update freq} & Low $\Rightarrow$ models differ & High $\Rightarrow$ models similar \\
\hline
\end{tabular}

\vspace{3mm}
\textbf{Rule of thumb:}
\begin{itemize}
    \item High-frequency updates ($\Delta t$ small) $\Rightarrow$ models similar
    \item Low-frequency updates ($\Delta t$ large) $\Rightarrow$ odometry more accurate
\end{itemize}

\end{tcolorbox}

% ============================================================================
\section{Implementation Notes}
% ============================================================================

\subsection{Common Pitfalls and Best Practices}

\begin{tcolorbox}[colback=red!5!white,colframe=red!75!black,title=\textbf{EXAM CRITICAL: Implementation Issues}]

\textbf{1. Angle Wrapping}
\begin{itemize}
    \item Always normalize angles to $[-\pi, \pi]$ or $[0, 2\pi)$
    \item Especially important in odometry model for $\delta_{rot1}$ and $\delta_{rot2}$
    \item Use: \texttt{atan2(y, x)} not \texttt{atan(y/x)} (handles quadrants correctly)
\end{itemize}

\textbf{2. Division by Zero}
\begin{itemize}
    \item In velocity model: $\omega = 0 \Rightarrow$ straight line motion
    \item Handle separately: $x' = x + v\cos\theta \cdot \Delta t$, $y' = y + v\sin\theta \cdot \Delta t$
\end{itemize}

\textbf{3. Coordinate Frames}
\begin{itemize}
    \item \textbf{Velocity model:} All in global frame
    \item \textbf{Odometry model:} Extract relative motion (frame-independent), then apply to global poses
\end{itemize}

\textbf{4. Noise Parameters}
\begin{itemize}
    \item Must be determined experimentally for each robot
    \item Larger values = more conservative (spread out distribution)
    \item Different surfaces $\Rightarrow$ different parameters (e.g., carpet vs. tile)
\end{itemize}

\textbf{5. Probability Function}
\begin{itemize}
    \item Can use Gaussian: $\text{prob}(a, b^2) = \frac{1}{\sqrt{2\pi b^2}}\exp\{-\frac{a^2}{2b^2}\}$
    \item Or triangular distribution for computational efficiency
    \item Must integrate/sum to 1 over all hypotheses
\end{itemize}

\end{tcolorbox}

\subsection{Parameter Tuning Guidelines}

\textbf{Velocity model parameters $(\alpha_1, \ldots, \alpha_6)$:}
\begin{itemize}
    \item Start with all $\alpha_i = 0.01$ (1\% error)
    \item Increase $\alpha_1, \alpha_2$ if robot struggles with straight-line accuracy
    \item Increase $\alpha_3, \alpha_4$ if robot struggles with turning accuracy
    \item Increase $\alpha_5, \alpha_6$ if systematic drift observed
\end{itemize}

\textbf{Odometry model parameters $(\alpha_1, \ldots, \alpha_4)$:}
\begin{itemize}
    \item Typical values: $\alpha_1 = 0.001$, $\alpha_2 = 0.001$, $\alpha_3 = 0.01$, $\alpha_4 = 0.01$
    \item Translation often more accurate than rotation
    \item Test on known trajectories (e.g., square, circle) and adjust
\end{itemize}

% ============================================================================
\section{Connection to Lab 4}
% ============================================================================

\begin{tcolorbox}[colback=yellow!5!white,colframe=orange!75!black,title=Your Lab 4 Implementation]

Your Lab 4 EKF uses a \textbf{hybrid approach}:

\vspace{2mm}
\textbf{Dynamics (prediction):}
\begin{itemize}
    \item Velocity-based with first-order model: $v_t = av_{t-1} + (1-a)u_v$
    \item More sophisticated than basic velocity model (includes dynamics)
\end{itemize}

\textbf{Measurements (correction):}
\begin{itemize}
    \item Wheel encoders: directly measure $\omega_r, \omega_l$
    \item IMU gyroscope: directly measure $\omega_g$
    \item Related to velocities via kinematics (like odometry uses encoder data)
\end{itemize}

\textbf{Key difference from textbook models:}
\begin{itemize}
    \item Your lab measures velocities directly (encoders + IMU)
    \item Textbook odometry integrates encoders into pose estimates
    \item Your approach: fuses raw sensor data optimally via EKF
\end{itemize}

\end{tcolorbox}

% ============================================================================
\section{Exam Preparation Summary}
% ============================================================================

\begin{tcolorbox}[colback=red!10!white,colframe=red!75!black,title=\textbf{What to Memorize for Exam}]

\textbf{1. Key Formulas:}
\begin{itemize}
    \item Velocity model: circular motion radius $r = |v/\omega|$
    \item Odometry decomposition: $\delta_{rot1}$, $\delta_{trans}$, $\delta_{rot2}$
    \item Noise model: variance proportional to squared velocities/motions
\end{itemize}

\textbf{2. When to Use Which Model:}
\begin{itemize}
    \item Velocity: planning (before motion)
    \item Odometry: estimation (after motion)
\end{itemize}

\textbf{3. Algorithm Structure:}
\begin{itemize}
    \item Both: compute ideal motion $\rightarrow$ compare to actual $\rightarrow$ return probability
    \item Independence assumption: $p_{total} = p_1 \cdot p_2 \cdot p_3$
\end{itemize}

\textbf{4. Implementation Issues:}
\begin{itemize}
    \item Angle normalization (critical!)
    \item Division by zero in velocity model
    \item \texttt{atan2} vs \texttt{atan}
\end{itemize}

\textbf{5. Noise Parameters:}
\begin{itemize}
    \item Know what each $\alpha_i$ represents
    \item Larger values = more uncertainty in that component
\end{itemize}

\end{tcolorbox}

% ============================================================================
% END OF ROBOT MOTION NOTES
% ============================================================================
