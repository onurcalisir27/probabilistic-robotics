\section{Robot Pose and Covariance Uncertainty Transformations}

\subsection{Definitions}

\subsubsection{Robot Position}
\begin{itemize}
  \item Displacement vector relative to some frame of reference, in Euclidean space.
  \item Pick one refernce point on the robot (body center), then a set of (x,y,z) coordinates of that point fully describes the robot position.
\end{itemize}
This describes the robots body position: $\mathbf{p} + \mathbf{\Delta p}$
,where $\mathbf{p} = \begin{bmatrix} x \\ y \end{bmatrix} \ \mathbf{\Delta p} = \begin{bmatrix} \Delta x \\ \Delta y \end{bmatrix} \ \mathbf{\Delta p}: \mathcal{N}(0, \Sigma_{xy})$

\subsubsection{Robot Pose}
\begin{itemize}
  \item Robot position and orientation, where body yaw is described as $\theta + \Delta \theta \ \Delta \theta: \mathcal{N}(0, \sigma_{\theta}^2)$
  \item Place a coordinate system on the robot, then its a matter of origin translation and coordinate system rotation
  \item Ways to represent:\ Position+Euler Angles, \ Position+Quaternions, \  Homogenous Transform Matrix
  \item And pose uncertainty: \ 1. Pose + Covariance Matrix \ 2. Pose Cloud
\end{itemize}

Pose and pose covariance is then given by: $\mathbf{P} = \begin{bsmallmatrix} \mathbf{p} \\ \theta \end{bsmallmatrix} \ \Sigma_{\mathbf{P}} =
\begin{bmatrix}
\sigma_x^2 & \sigma_{xy} & 0 \\
\sigma_{xy} & \sigma_y^2 & 0 \\
0 & 0 & \sigma_{\theta}^2
\end{bmatrix}$

\subsubsection{Transformations}
\begin{itemize}
  \item Transformations from local (robot) frame of reference to the global (map) frame of reference is the robot pose.
  \item Transformation is the displacement needed to bring the global frame into alignment with the local frame.
  \item Transformation is the conversion of point (or vector) coordiantes in the local frame to global frame.
\end{itemize}

\begin{equation}
  \mathbf{T_{RA}} =
  \begin{bmatrix}
      \mathbf{R}_{3x3} & \mathbf{t}_{3x1} \\
      \mathbf{0}_{1x3} & 1
  \end{bmatrix}
  \label{eq:Homogenous Transform matrix}
\end{equation}

where \textbf{R} is the reference frame, and \textbf{A} is the local frame.
But why should we care about transforms?

\begin{figure}[H]
  \begin{center}
    \includegraphics[width=0.5\textwidth]{images/transforms.png}
  \end{center}
  \caption{Robot Transformations Chain}\label{fig:Transforms}
\end{figure}

\subsection{Uncertainty Propagation (2D case)}
\subsubsection{Pose Uncertainty}
Uncetain robot pose in map reference frame:
\begin{equation}
  \mathbf{T}_{MR} = \begin{bmatrix} \mathbf{R}(\theta + \Delta \theta) & \mathbf{p}+\mathbf{\Delta p} \\ \mathbf{0}_{1x2} & 1 \end{bmatrix}_{3x3}
\end{equation}
Uncertain camera pose in map reference frame:
\begin{align}
  \mathbf{T}_{MC} &= \mathbf{T}_{MR} \mathbf{T}_{RC} \\
  \mathbf{T}_{RC} &= \begin{bsmallmatrix} \mathbf{R}(\phi) & \mathbf{t} \\ \mathbf{0} & 1 \end{bsmallmatrix}
\end{align}
\begin{figure}[H]
  \begin{center}
    \includegraphics[width=0.4\textwidth]{images/transform_uncertainty.png}
  \end{center}
  \caption{Visualizing Uncertainty}\label{fig:Visualizing Uncertainty}
\end{figure}

Expanding the notation
\begin{align*}
  \mathbf{T}_{MC} &= \mathbf{T}_{MR} \mathbf{T}_{RC} \\
  \mathbf{T}_{MC} & = \begin{bmatrix} \mathbf{R}(\theta + \Delta \theta) & \mathbf{p}+\mathbf{\Delta p} \\ \mathbf{0} & 1 \end{bmatrix} \begin{bmatrix} \mathbf{R}(\psi) & \mathbf{t} \\ \mathbf{0} & 1 \end{bmatrix} \\
  \mathbf{T}_{MC} & = \begin{bmatrix} \mathbf{R}(\theta + \psi + \Delta \theta) & \mathbf{R}(\theta + \Delta \theta)\mathbf{t} + \mathbf{p}+\mathbf{\Delta p} \\ \mathbf{0} & 1 \end{bmatrix}
\end{align*}

We can see that, the rotation uncertainty does not change, however we have added a new term to the position in the form of $ \mathbf{R}(\theta + \Delta \theta)\mathbf{t}$ which forms the "banana shape". The original position uncertainty also remained.\\

Linearizing:
\begin{align*}
  \mathbf{n} &= \mathbf{R}(\theta + \Delta \theta)\mathbf{t} \\
  \mathbf{n} & = \begin{bmatrix}n_x \\ n_y \end{bmatrix} =  \begin{bmatrix}t_x\cos(\theta + \Delta \theta) - t_y \sin(\theta + \Delta \theta) \\ t_x\sin(\theta + \Delta \theta) + t_y \cos(\theta + \Delta \theta) \end{bmatrix} \\
            &  \text{Using small angle approximations} \\
  \mathbf{n} & = \begin{bmatrix}n_x \\ n_y \end{bmatrix} =  \mathbf{R}(\theta) \begin{bmatrix} t_x \\ t_y \end{bmatrix} - \mathbf{R}(\theta -\frac{\pi}{2}) \begin{bmatrix} t_x \\ t_y \end{bmatrix} \Delta \theta\\
\end{align*}

Putting it together, we get:
\begin{equation}
  \mathbf{T}_{MC} = \begin{bmatrix} \mathbf{R}(\theta + \phi + \Delta \theta) & \mathbf{R}(\theta) \mathbf{t} + \mathbf{p} - \mathbf{R}(\theta - \frac{\pi}{2})\mathbf{t} \Delta \theta + \mathbf{\Delta p}) \\
                  \mathbf{0} & 1
  \end{bmatrix}
  \label{eq:position uncertainty}
\end{equation}
\\
\begin{itemize}
  \item Position uncertainty has two components:
  \begin{itemize}
    \item Original uncertainty characterized by $\Sigma_{xy}$.
    \item Orientation uncertainty characterized by $\sigma_{\theta}^2$, projected to x/y axes
  \end{itemize}
  \item Result is a banana-shaped point-cloud approximated by an ellipse, which stretches the original ellipse in tangent direction
\end{itemize}

For Covariance, we assume independent position and orientation estimates. We will also make use of the corrollary that covariance terms are additve. Then, camera pose and covariance can be written as:
\begin{align*}
  Q &= \begin{bsmallmatrix} \mathbf{R}(\theta)\mathbf{t} + \mathbf{p} \\ \theta + \phi \end{bsmallmatrix} \\
  \Sigma_Q &= \begin{bmatrix} \Sigma_{xy} + \Sigma_{n} & 0 \\ 0 & \sigma_{\theta}^2 \end{bmatrix} \\
  \Sigma_n &= \mathbf{R}(\theta - \frac{\pi}{2})\mathbf{t}\sigma_{\theta}^2 \left( \mathbf{R}(\theta - \frac{\pi}{2})\mathbf{t}\right)^T\\
  \Sigma_n &= \mathbf{R}(\theta - \frac{\pi}{2}) \ \mathbf{t} \mathbf{t}^T \ \mathbf{R}(\theta - \frac{\pi}{2})\sigma_{\theta}^2\\
\end{align*}
\begin{figure}[H]
  \begin{center}
    \includegraphics[width=0.3\textwidth]{images/banana_cov.png}
  \end{center}
  \caption{Covariance Uncertainty Propagation}\label{fig:}
\end{figure}

\subsubsection{Orientation Uncertainty}
Orientation uncertainty is one-dimensional in the 2D plane (planar robots), consisting of yaw, ($\theta$) and yaw variance $\sigma_{\theta}^2$.
Gaussian approximation is valid for small errors in yaw however, one must be careful with wraparound as yaw needs to be bounded by $2\pi$.

So how do we represent uncertainty in 3D?
\begin{align*}
R_{x,\theta} &= \begin{bmatrix}
1 & 0 & 0 \\
0 & \cos\theta & -\sin\theta \\
0 & \sin\theta & \cos\theta
\end{bmatrix} \\[1em]
R_{y,\theta} &= \begin{bmatrix}
\cos\theta & 0 & \sin\theta \\
0 & 1 & 0 \\
-\sin\theta & 0 & \cos\theta
\end{bmatrix} \\[1em]
R_{z,\theta} &= \begin{bmatrix}
\cos\theta & -\sin\theta & 0 \\
\sin\theta & \cos\theta & 0 \\
0 & 0 & 1
\end{bmatrix}
\end{align*}

\newpage
\subsection{General 3D Rotation}

A straightforward approach would be to compose basic rotations: Euler angles, yaw-pitch-roll, where the order matters and the choise of rotation of stationary frame.
A set of rotation matrices and matrix multiplication form a group that we call \textbf{SO(3) group}.\\

\subsubsection{Axis-Angle Representation}
Based on Euler Rotation Theorem:

\begin{figure}[H]
  \begin{center}
    \includegraphics[width=0.7\textwidth]{images/axisangle.png}
  \end{center}
  \caption{Euler Angle Representation}\label{fig:Euler Angles}
\end{figure}

\subsubsection{Unit Quaternions}
Quaternions are an extension of complex numbers, which represent algebra on axis-angle representation.
The quaternion product is a composition of rotations, which also forms an SO(3) group.

\begin{align*}
  \mathbf{q} &= \cos \frac{\theta}{2} + \sin \frac{\theta}{2} (k_x \mathbf{i} + k_y \mathbf{j} + k_z \mathbf{k}) \\
  \mathbf{i}^2 &= \mathbf{j}^2 = \mathbf{k}^2 = \mathbf{ijk} = -1
\end{align*}

\subsubsection{Rotation Matrices}
Properties of rotation matrices
\begin{itemize}
  \item multiplication is associative, but not commutative
  \item it is commutative in 2D, SO(2) group
  \item $\det (\mathbf{R}) = 1$ (for right-hand coordinate systems)
  \item Columns are orthogonal unit vectors
  \item $\mathbf{R}^{-1} = \mathbf{R}^T$
\end{itemize}

\subsubsection{Angular Velocity}
\begin{figure}[H]
  \begin{center}
    \includegraphics[width=0.3\textwidth]{images/angular_vel.png}
  \end{center}
  \label{fig:Angular velocity}
\end{figure}

Which way are the x, y, and z vectors spinning?
\begin{align*}
  \dot{\bf{x}} &= \bf{k}\omega \times \bf{x} \\
  \dot{\bf{y}} &= \bf{k}\omega \times \bf{y} \\
  \dot{\bf{z}} &= \bf{k}\omega \times \bf{z}
\end{align*}

What about projections of the axes of the rotating frame?
\begin{align*}
  \bf{R} &= [\bf{r}_1 \ \ \bf{r}_2 \ \ \bf{r}_3] \\
  \bf{\dot{R}} & = [\bf{\dot{r}}_1 \ \ \bf{\dot{r}}_2 \ \ \bf{\dot{r}}_3] \\
  \bf{\omega}_k &= \bf{k} \omega \\
  \bf{\dot{R}} &= \bf{\omega}_k \times \bf{R}
\end{align*}

\subsubsection{Exponential Coordinates}
To get rid of the cross product
\begin{align*}
  \dot{\mathbf{R}} & = \omega_k \times \mathbf{R} \\
  \dot{\mathbf{R}} & = \lfloor \omega_k \rfloor \times \mathbf{R} \\
  \lfloor \omega_k \rfloor &= \begin{bmatrix} 0 & -\omega_z & \omega_y \\ \omega_z & 0 & \omega_x \\ -\omega_y & \omega_x & 0\end{bmatrix}
\end{align*}

Formulate this problem:
Frame is rotating at angular velocity $\omega$ around $k$.
Its initial orientation is described by $R(0)$.
What will its orientation $R(t)$ be after time $t$? \\

Answer? Solve the above differential equation!
\newpage
\subsubsection{Exponential Mapping}
Solution to $ \dot{\mathbf{R}} = \lfloor \omega_k \rfloor \times \mathbf{R}$ is:
\begin{equation}
  {\mathbf{R}}(t) = e^{\lfloor \omega_k \rfloor t} \ \mathbf{R}(0) = e^{\lfloor \theta_k(t) \rfloor }\mathbf{R}(0)
  \label{eq:exponential mapping}
\end{equation}

What is $\exp\left\{{\lfloor \omega_k \rfloor t}\right\}$? We can use Taylor expansion to figure it out, by using convenient property that $\lfloor k \rfloor ^3 = - \lfloor k \rfloor$ (k is a unit vector).
Get the Rodrigues formula:
\begin{equation}
  e^{\lfloor \theta_k(t) \rfloor } = I + \lfloor k \rfloor \sin \theta + \lfloor k \rfloor ^2 (1-\cos \theta)
  \label{eq:Rodrigues formula}
\end{equation}

\begin{itemize}
  \item Frame orientation at time 0 is described by $\mathbf{R}(0)$
  \item Frame is rotating around axis $\mathbf{k}$ in \textbf{global} frame at rate $\omega$
  \item Vector $\omega_k = \omega \mathbf{k}$ is called \textbf{exponential coordinates}.
  \item Why? Because it is used in exponential mapping $e^{\lfloor \omega_k \rfloor }$
  \item The operator $\lfloor x \rfloor$ is a skew-symmetric matrix of vector x.
  \item Angular velocity integrates in exponential coordinate space, $\lfloor \theta_k(t) \rfloor = \lfloor \omega_k \rfloor t$, before mapping!
  \item We can use the Rodrigues formula to calculate the mapping.
  \item So at time $t$, we have $\mathbf{R}(t) = e^{\lfloor \theta_k(t) \rfloor} \mathbf{R}(0)$
  \item Rotation is around $k$ in fixed frame so multiplication is on the left!
\end{itemize}

\subsubsection{Covariance of Rotations}

What do the rows and columns in covariance matrix represent?\\
Variance and covariance of exponential coordinates tangential to the rotation.\\
We are looking at the point in SO(3) space but expressing its variations in so(3) space.\\

\textbf{Covariance after rotation:}
\begin{align*}
  \tilde{R}_{mb} &= D_m R_{mb} \\
  \tilde{R}_{mb} &= e^{\lfloor \delta \rfloor} R_{mb} \\
  R_{wm} \tilde{R}_{mb} &= R_{wm}e^{\lfloor \delta \rfloor} R_{mb} \\
  \tilde{R}_{mb} &= R_{wm}e^{\lfloor \delta \rfloor} R_{wm}^{-1} R_{wm} R_{mb} \\
  \tilde{R}_{mb} &= R_{wm}e^{\lfloor \delta \rfloor} R_{wm}^{-1} R_{wb} \\
  \tilde{R}_{mb} &= e^{\lfloor R_{wm}\delta \rfloor} R_{wb} \\
\end{align*}

Exponential coordinates of the disturbance have been rotated!\\

\textbf{Rotation Inversion:}
\begin{align*}
  \tilde{R}_{mb} &= e^{\lfloor \delta \rfloor} R_{mb} \\
  \tilde{R}_{mb}^{-1} &= R_{mb}^{-1} e^{\lfloor -\delta \rfloor}  \\
  \tilde{R}_{mb}^{-1} &= R_{mb}^{-1} e^{\lfloor -\delta \rfloor} R_{mb} R_{mb}^{-1} \\
  \tilde{R}_{mb}^{-1} &= e^{\lfloor -R_{mb}^T \delta \rfloor} R_{mb}^{-1} \\
\end{align*}

Covariance is quadratic, so negative sign does not change it.
Hence $\Sigma_{inv} = R_{mb}^T \Sigma R_{mb}$

\begin{itemize}
  \item Have uncertain rotation $R_1$ with covariance $\Sigma_1$
  \item Apply deterministic rotation $R$ in global frame
  \item Composite rotation is $RR_1$ with covariance $R\Sigma_1 R^T$
  \item If rotation is in local frame, covariance is unchanged
  \item If both rotations are uncertain, $(R_1, \Sigma_1)$ and $(R, \Sigma)$ and we rotate $R_1$ by $R$ in global frame, the resulting covariance is
    $\Sigma + R \Sigma_1 R^T$
  \item Inversion expresion looks similar
\end{itemize}


